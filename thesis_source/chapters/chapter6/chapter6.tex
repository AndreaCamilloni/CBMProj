\graphicspath{{chapters/chapter6/}}
\chapter{Conclusion} \label{chapter6}
The black-box nature of deep learning models is one of the main issues in their adaption in high risk real-world settings. As a matter of fact, as humans we tend to arrive to conclusions through a certain thought process, which acts as an explanation to our final answer. Understanding "why" a certain answer is given, generally helps to increase the trust towards the decision.
Possible solutions to further understand how black-box models makes their decisions have been studied in the literature. In the framework of concept bottleneck models, the models are trained to learn a set of interpretable concepts over which the final and main prediction is made.\\
This framework is particularly suitable for the problem of automated diagnosis of melanoma, the most widespread form of skin cancer. As a matter of fact, in clinical practice it is common to identify a series of attributes as an explanation for the final diagnosis. The 7-point checklist ia a well-known clinical rule-based method to predict melanoma based on the presence of irregularity in seven attributes.
In this work, the CBM framework has been applied to this problem using the publicly available 7pt-derm dataset. Two of the CBM implementations presented in the literature have been replicated, and different approaches have also been proposed for the implementation of these two architectures, the independent and the sequential.
The 2 CBM architectures replicated using the $ IncNet_ {MTL} $ as an intermediate model, were the best in this framework, since the $ IncNet_ {MTL} $ provides more accurate concepts than the other MTL architectures; the intermediate model $ IncNet_ {MTL} $ having been trained on the Derm7pt dataset has in fact shown the best performance in the prediction of concepts compared to the other 2 implementations. \\
The experiments have shown that the performance of the end-to-end black-box models is slightly superior. However, they lose the ability to provide further insights into prediction as CBM can do.\\\\
In addition to Independent and Sequential configuration, the Joint bottleneck configuration was also proposed in \cite{CBM}.In the Joint bottleneck approach, the concept prediction model(s) and the model that performs the main classification task are trained at the same time in end-to-end fashion. Furthermore, in \cite{CBM} the Joint model shows better performance than Independent and Sequential bottleneck on general images. In a future work, experiments on Joint Bottleneck model could be carried out as a means to improve the results of the current approach.