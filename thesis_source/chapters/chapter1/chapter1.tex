\graphicspath{{chapters/chapter1/}}
\chapter{Introduction} \label{introduction}
Skin cancer is the most common malignancy in fairskinned populations, and the incidences of melanoma and non-melanoma skin cancers have been rising in recent years~\cite{7ptCNNforMel}. If undetected these malignancies have high death rate, but early diagnosis of melanoma has been shown to reverse the odds in the majority of cases~\cite{mtl7ptCoppola}. \\
Skin cancer detection seems to be accurate when performed by dermatologist using a dermatoscope, which is a handheld instrument that permits in vivo evaluation of colors and microstructures of the skin that are not visible to the naked eye. In the literature some algorithms for melanoma diagnosis have been studied, and one of them is the 7 point checklist \cite{Kawahara}, which consists in looking for irregular patterns and assigning a score to them: melanoma is diagnosed if a score greater or equal than 3 is achieved~\cite{Kawahara}. \\
Deep learning methods show great performance in melanoma diagnosis and they could be an important tool in medical applications. However, in a real-world scenario, an interpretation of how the model predict the diagnosis, is a need for dermatologists to be sure about the final diagnosis; most state-of-the-art models today do not typically give an explanation about theirs predictions, because they are end-to-end models that go from raw input $x$(e.g. image) to target $y$(e.g. melanoma diagnosis).\\
So in this work, several implementations of end-to-end models were compared with concept bottleneck models(CBM)\cite{CBM} for prediction of melanoma diagnosis in skin lesion images. \\
CBM allow to approach the problem of explaining how the model makes its predictions by revisiting the idea of first predicting an intermediate set of human-specified concepts like "atypical pigment network", then using them to predict the target $y$(Melanoma or Nevus). \\
In this work, experiments were carried out on different approaches, from the most widespread,the Single Task Learning, to the implementation of CBMs.
In the case of Single Task Learning, different methods were presented, based on the use of pre-trained networks, respectively InceptionV3\cite{IncNet} and ResNet101V2\cite{Resnet}; furthermore, the performance of these models was tested both by freezing the pre-trained weights and fine-tuning them to the problem at hand.
With regard to the CBM implementations, pre-trained Multi Task Learning architectures have been used as intermediate models for the concepts prediction; whereas logistic regression model was used for the final classification. For these experiments, the independent and sequential configurations described in \cite{CBM} have been implemented.\\
This manuscript begins with a brief overview of medical and technical background, followed by proposed works in the literature that deal with automated diagnosis of skin lesions. This is followed by a description of the 7-point derm dataset and the methods that have been used for the experiments. The last chapters presents the experimental setups and results obtained in the experiments as well as a conclusion to the work.

